\documentclass[titlepage]{article}
\usepackage[bottom=4cm, right=4cm, left=4cm, top=4cm]{geometry}
\usepackage{graphicx}
\usepackage{hyperref}
\usepackage{float}
\restylefloat{table}
\usepackage{amsmath}
\usepackage{booktabs}

\title{SFWRENG 4NL3\\ Project Step 3 - Annotation Task Report}
\author{Sumanya Gulati}
\date{March 2025}

\begin{document}

\begin{titlepage}
    \maketitle
\end{titlepage}

\newpage

\section{Description of Task}
% A brief description of the task, as you understand it.
The task consisted of labelling the  messages sent by users right before or while playing
the DOTA game. The conversations were broken down into games which in my understanding 
corresponded to a single group session within DOTA. The annotator was asked to label each message 
in a game's conversation with a number from the range 0-7 categorizing the messages based on 
their content and apparent sentiment.

\section{Interesting Aspects}
% What was most interesting to you about the task and the data?
Despite the short length of each message (usually 2-4 words each), given the context of the entire
conversation, I could infer a lot more information about the players and their relationships with 
each other than I expected. It was easy to guess which group of players in a session were acquaintances
based on the way they interacted with each other or by the number of \emph{casual} messages exchanged 
between them. 

\section{Insights and Challenges}
% What did you learn about the task from doing the annotation? What challenges do you
% expect models to face? What surprising things did you observe in the data? Which features
% do you expect to be useful?
I am unsure about what the other datasets look like but if they are similar to Part 4, I believe the model 
might not be trained over diversified content. This is mostly because out of all the messages I annotated, a 
majority belonged to this set of words - `ez', `gg', `hahaha', `lol' or their variations. For the model to 
generate actual, meaningful results, it must be trained on more expressive test messages that are representative 
of all the 7 labels.\\
\newline
Additionally, I think the preprocessing of the data will require a lot more nuance because a majority of the 
text messages had typos in them. This presents two options - either disregard any message with a typo in it which 
would reduce the dataset by a considerable amount or, label them based on their inferred context but have a 
robust preprocessing system set up. Even if the team chooses the latter option, I am not sure if a sufficiently 
broad enough system can be developed to accurately identify the typos and correct them unless done manually.


\section{Mental Model}
% Did you come up with your own mental model of how to do the task and can you (at a high
% level) describe that?
Very quickly into the task, I came to the realization that most messages were either `ez', `gg' or some variation 
of that which would be labelled as \emph{positive}. I also put anything with a curse word directly into the \emph{verbal abuse}
category which further simplified the task. I had chosen to label test-based emojis as \emph{non-English} and all 
variations of `lol', `lmao', `haha' as \emph{miscellaneous}. Since about three-quarters of the messages fell neatly 
into one of these categories, it was pretty easy to annotate them and I made my way through the dataset in half the allocated time.

\section{Ambiguity in Instructions}
% Was there anything unclear about the instructions? How would you recommend modifying
% them in the future to make them more helpful?
The instructions did not specify the order of priority for messages that could be labelled as two or more of the given 
categories. For example, if an otherwise positive message also contains the word `haha' or some other random sequence of 
characters, should it be labelled as \emph{positive} or \emph{non-English}? Leaving the order of precedence up to the annotators
might cause discrepancy in the annotator agreement ratios.\\
\newline
Furthermore, for a lot of the messages, longer sentences were broken down across multiple messages, it would have been helpful to 
have clear instructions about whether these should be labelled based on the messages that follow and provide additional context or 
without taking those into consideration. For example, in one of the games the sequence of messages was as follows - `apparently' which 
could either be \emph{casual} or \emph{cooperative} based on what the following messages are, and then a message describing the game 
strategy. 

\end{document}