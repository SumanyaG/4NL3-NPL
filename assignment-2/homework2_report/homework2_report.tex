\documentclass[titlepage]{article}
\usepackage[bottom=3cm, right=2cm, left=2cm, top=3cm]{geometry}
\usepackage{graphicx}
\usepackage{hyperref}
\usepackage{float}

\title{SFWRENG 4NL3 Assignment 2}
\author{Sumanya Gulati}
\date{4 February 2025}

\begin{document}

\begin{titlepage}
    \maketitle
\end{titlepage}

\newpage 

\tableofcontents
\listoftables
\listoffigures

\newpage

\section{Dataset}
% Describe the dataset you chose and why you think it is interesting or useful to analyze.

\subsection{Data Collection and Splitting}
% How did you collect the data, choose the categories, and split it into documents?

\subsection{Description}
% Include tables and figures to show the size of your dataset, e.g. a table with the number of
% documents and the average number of tokens per document, broken down by category.
 
\section{Methodology}
% Describe the steps that you performed and what informed your decisions.
% For example, if you decided not to lowercase the text because it gave better results at a later
% stage, include that decision and your reasoning for doing that. This should include details
% of your preprocessing steps, e.g. lowercasing, stemming, not just saying “each document was
% preprocessed”. Describe the kind of analysis you performed. For any steps that you did not
% implement yourself (e.g. topic modeling), you should mention which package/library was
% used.

\section{Results and Analysis}
% Present your results as formatted tables and figures. You should
% have at least one table or figure for each of 2.3, 2.4, and 2.5. This must include at least the
% results of the required steps, but may also include any interesting findings you came across
% (e.g. results of topic modeling with and without a given preprocessing step that made a
% difference in the quality of the results). For each table and figure, include a description of
% your main takeaways.

\section{Discussion}

\subsection{Findings}
% Include two subsections in the discussion. The first should talk about what
% you learned about your dataset. Imagine that you are describing what your results showed,
% at a high level, to a friend who does not have any NLP experience but is interested in the
% corpus that you chose.

\subsection{Reflection}
% The second subsection should cover what lessons you personally
% learned during the completion of the assignment. You might write about how you found and
% processed the data, preprocessing effects on downstream analysis, topic modeling results,
% limitations of your approaches, or other interesting aspects that were new to you.

\section{Appendix}

\end{document}